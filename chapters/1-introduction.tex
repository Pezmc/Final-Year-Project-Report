\chapter{Introduction}

\begin{comment}
This chapter puts the work into context. Having read it, the reader should be left in no doubt as to:

- the topic area to which the work applies
- why the work is being done
- what else has been done in the area and by whom
 - how the author proposes to tackle the problem: The project proposal is often expressed in terms of a main objective and possibly one or more additional objectives. It is useful to define "milestones" or "sub-goals" that mark the progress towards the objectives. 
 - It is common to end this chapter with a brief overview of each of the subsequent chapters of the report.
\end{comment}

Traditionally the management of personal finances is performed by viewing bank statements provided by the users bank, however, even in this modern age of 'internet banking' banks offer a limited set of tooks that mimic the paper statements that would have been sent in the past.

This project sets out to build an online application that can be used to manage personal finances. There are two main parts of the project; firstly users can upload bank statements, which are displayed and navigated in an intuitive manner; secondly, once the application has enough historical data, predicting the users future outflow.

\section{Motivation}
% \plan{Help make managing finances easier, basic survey of people + appendix with details, any relevant paperts}
There are four main steps when producing and using a budget, recording previous expenses, sorting these into categories
using this historical information to guess future expenditure and evaluate accuracy of these predictions and using new information to update them.

Since the liquidity crisis of 2009 \cite{gore2010}, budgets have been squeezed and the average persons personal disposable income has fallen, hitting a nine-year low in 2012\cite{barnard2012households}. With experts suggesting that ``Budgets are essential for financial planning''\cite{wsj2013budget}, research suggesting that personal budgets lead to a ``positive impact on ``mental wellbeing''\cite{tlap2013budget} and guides from UCAS, the UoM SU Advice Centre and The Manchester University Crucial Guide encouraging use of budgeting, it is clear that producing a budget is of benefit.

However, in an informal survey\footnote{Appendix \ref{app:budgetsurvey}} the majority of students surveyed, did not make a budget. An easier way to produce a budget can hopefully increase the amount of students relying on one.

Increasing use of debit cards\cite{bbc2010debit} means that bank statements contain more and more information about where people spend their money. With access to those bank statements now provided online, with most UK banks offering the option to export \gls{transaction} history, individual users can collate a database of their personal spending habits.

The increasing availability of this data, combined with more detailed transaction history makes it possible to automate the four main steps of producing a budget, and this is the idea behind the project.

\section{Aims and Objectives}
%\plan{Things I set out to do, designed by talking to people to get an idea of features they would like}

The key objectives of this project can be split into three parts, the management of statements, making predictions using those statements and ensuring a high level of security.

\subsection{Statement Management}
Implement an intuitive way to view and manage personal finances.
There are several key parts to this, upload and parsing of transactions from statements downloaded from a bank, resolving the references found on the statement to the transactor they represent and categorising the transactions.
    
\subsection{Predictions}
Make predictions of future transactions that a user will make using a model that is fitted to their spending behaviour, the application will need to predict whether or not spending will occur and how much money will be spent.
Using

\subsection{Security}
As the application will deal with information of a sensitive nature strong security techniques are particularly important.
The project takes this into account, considering possible attack vectors and taking steps to mitigate those attacks.

\section{Overview of Report}
\plan{This report covers some of the key design decisions, implementation decisions and then what the application does}

Chapter \ref{cha:background}: Background 
An review of existing products in the market and an introducion to techniques used in the project.