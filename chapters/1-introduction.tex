\chapter{Introduction}

\begin{comment}
This chapter puts the work into context. Having read it, the reader should be left in no doubt as to:

- the topic area to which the work applies
- why the work is being done
- what else has been done in the area and by whom
 - how the author proposes to tackle the problem: The project proposal is often expressed in terms of a main objective and possibly one or more additional objectives. It is useful to define "milestones" or "sub-goals" that mark the progress towards the objectives. 
 - It is common to end this chapter with a brief overview of each of the subsequent chapters of the report.
\end{comment}

Traditionally the management of personal finances is performed by viewing bank statements provided by the users bank, however, even in this modern age of 'internet banking' banks offer a limited set of tooks that mimic the paper statements that would have been sent in the past.

This project sets out to build an online application that can be used to manage personal finances. There are two main parts of the project; firstly Users can upload  bank statements, which are displayed and navigated in an intuitive manner; secondly, once the application has enough historical data, predicting the users future outflow.

\section{Motivation}
\plan{Help make managing finances easier, basic survey of people + appendix with details, any relevant paperts}


Since the liquidity crisis of 2009 \cite{gore2010}, budgets have been squeezed and the average persons personal disposable income was at a nine-year low in Q1 2012\cite{barnard2012}. With experts suggesting that ``Budgets are essential for financial planning''\cite{wsj2013budget}, research suggesting that personal budgets lead to a ``positive impact on ``mental wellbeing''\cite{tlap2013budget} and guides from UCAS, the UoM SU Advice Centre and The Manchester University Crucial Guide encouraging use of budgeting, an easier way to product a budget is of clear benefit.

Increasing use of debit cards\cite{bbc2010debit} means that bank statements contain more and more information about where people spend their money. With access to those bank statements now provided online, with most UK banks offering the option to export \gls{transaction} history, individual users can collate a database of their personal spending habits.

The increasing availability of this data, combined with more detailed transaction history makes it possible to automate the four main steps of producing a budget, and this is the idea behind the project.

\section{Existing work in this area}
\plan{Other applications, and relevant research papers that I can find, similarity to stock market, predicting the future}

There are several other applications that attempt to implement the features similar to this project, most notably Lloyds TSB Money Manager \cite{lloyds2014money}, which was the first service provided by a bank in the UK to offer this style of money management application.

The service is available to Lloyds TSB current account holder and offers several key features

\begin{enumerate}
\item Categorising your spending
\item Creating spending plans per category.
%\item View dates money came in and out in a calendar
\item View money spent per category
\item Track your progress.
\end{enumerate}

\missingfigure{Figures of Lloyds money manager}
\plan{Mention the lloyds things}


\section{Aims and Objectives}
\plan{Things I set out to do, designed by talking to people to get an idea of features they would like}

\subsection{Parsing Statements}
\plan{Loading a statement into the actual system, do I mention named entity resolution here?}

\subsection{Predicting expenditure}
\plan{Predicting how much money you're going to spend in the future (or next month)}

\section{Overview of Report}
\plan{This report covers some of the key design decisions, implementation decisions and then what the application does}
