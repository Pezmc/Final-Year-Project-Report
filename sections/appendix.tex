\begin{appendices}
\chapter{Some Appendix}

\section{Survey} \label{app:budgetsurvey}
Informal survey of 12 CS students in the third year lab. 

Questions:
\begin{enumerate}
\item Do you currently make a budget?
\item Do you stick to that budget?
\item Do you find your budget has a `positive' impact?
\end{enumerate}

% Booktabs require to add  to your document preamble
\begin{table}[h]
\centering
\caption{Survey Results}
\begin{tabular}{@{}llll@{}}
\toprule
       & \multicolumn{3}{l}{Question} \\ 
Answer & 1.       & 2.      & 3.      \\ \midrule
yes    & 5        & 1       & 3       \\
no     & 7        & 4       & 4       \\
n/a    & 0        & 7       & 5       \\ \bottomrule
\end{tabular}
\end{table}

\begin{comment}
Appendix A, B, C, etc.
These appendices can be very useful for giving detail that would otherwise disrupt the flow and readability of the report. They are given titles (e.g. "Appendix A: Example of the operation of the system") and bound in with the report. In general they are optional though, by convention, for a programming project, Appendix A often contains a non-trivial illustrative example of an input to the system and the corresponding output. For some projects, appendices may include tables of data. However, very long tables of data (more than about 10 pages) should be relegated to the Auxiliary Material, and not submitted as part of the Final Report. Program listings (apart from short code snippets) should likewise not be submitted as part of the Final Report.
\end{comment}

\section{Hashing Test} \label{app:hashingtest}}

Implemented in PHP, test was run on a 2.7 Ghz Intel Core i7 with 16 Gb of 1600 Mhz DDR3 RAM.

\lstinputlisting[language=php]{code/passwordHashingTest.php}

\end{appendices}